\subsection{Role Descriptions and Features}

A common way to describe desirable features for a software project is to
formulate them in the form of user stories. Such a user story usually follows
the pattern shown in Figure~\ref{fig:user_story}.



\begin{figure}[H]
  \centering
  \begin{minted}{text}
    As a <Role>
    I want <Feature>
    So that I can have <Benefit>
  \end{minted}
  \caption{\label{fig:user_story} The standard pattern for a user story.}
\end{figure}



For the E-commerce website there are three kinds of users. Customers, visitors and
administrators. Below, a summary of the roles can be found. For complete user
stories, please refer to Appendix~\ref{app:user_stories}. Due to lack of time the user stories are not featured in the standard pattern manner.

\subsubsection{Role: Visitor}\label{sec:visitor_role}

A Visitor is any user that is not logged in. Such a visitor is able to list
products, view product information find their current availability. They are
also offered the option of registering an account on the site, thus becoming
a Customer.

\subsubsection{Role: Customer}\label{sec:user_role}

A Customer is a registered user \textbf{without} admin privileges. In order to
access Customer privileges it is required that the Customer is logged in.
A Customer can do everything that a Visitor can do, as well as adding items
to a shopping cart and making orders.

\subsubsection{Role: Admin}\label{sec:admin_role}

A Admin is a registered user \textbf{with} admin privileges. In order to access
Admin privileges it is required that the Admin is logged in. An Admin can do
everything that a Customer can do. An Admin can also add, update and remove
Products and Product Types, as well as handling orders.
