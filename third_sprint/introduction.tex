\section{Introduction}\label{sec:introduction}

The students were tasked with building a web application that was to be
used for e-commerce. The focus was to be placed on the database and server
layers, while constructing a usable website through which the user could
purchase items.

\subsection{Requirements}\label{sec:requirements}

The system should be able to process orders. There should be the concept
of a shopping cart to which a customer can add products, and then check out
with payment. It was strongly recommended that the database be a relational
one. The use of a ORM for creating schemas was disallowed, because the goal
of the assignment was to learn to administer the database.

\subsection{Choice of Merchandise}\label{sec:merchandize}

Multiple different suggestions for merchandise were discussed, including
computers and cars. Eventually it was decided, that because the merchandise
itself was \emph{not} the main focus of the project, something less obvious
might as well be chosen. It was decided to sell berries, and in order to
make the variety of items sold a bit more diverse (for a more interesting
project), it was also suggested that related supplies, such as filters and
jars for jam production, would also be sold.

\subsection{Technologies}\label{sec:technologies}

The choice of technologies was very open, which warranted some research into
different technologies. The choices were then made in favor of technologies
that would offer the most learning.

The choice for most of the stack fell on Django~\cite{django}, because,

\begin{enumerate}
  \item It is a widely used, tried and tested technology
  \item Knowing Django could be an advantage in a future career
  \item Other colleagues told us it would be difficult, due to the constraint
    about ORMs (See Section~\ref{sec:requirements}).
\end{enumerate}

It was decided to make the application as production ready as possible,
hosting it on a remote server inside docker~\cite{docker} containers. The
Django server was run using gunicorn~\cite{gunicorn}, which was then
mirrored using nginx~\cite{nginx} for added robustness.

The choice for the database fell upon PostgreSQL~\cite{postgres} because it
is a popular alternative to MySQL~\cite{MySQL} and has been gaining popularity
for many years. It is therefore a good technology to be acquainted with.
