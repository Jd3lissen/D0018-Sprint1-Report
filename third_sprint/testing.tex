\newpage

\subsection{Test-cases}

Support for automatic tests was added in the following steps. A test container
was added to docker. When that container runs, it performs all the tests
using pytest~\cite{pytest}. Code coverage analysis is then reported using
coverage.py~\cite{coverage}.

Support for Selenium~\cite{selenium} was also added, in order to support
end-to-end testing. Selenium relies on the browsers Firefox/Google Chrome,
and so far attempts to make those run in docker have been unsuccessful. There
are hopes, however, that such support could be implemented without much hassle.

Thus, though no actual test cases have been written, the groundwork was laid
for being able to perform automated tests.

In the absence of automatic tests, the software was tested manually and often.
This was done by comparing the information stored in the database with the
information displayed on the screen, and checking if it made sense. Every
once in a while, the list of finished tasks was gone through manually to
check that there were no regression on previously solved issues.
